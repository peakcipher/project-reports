\chapter{\label{method}Methods}

\setcounter{equation}{0}
\setcounter{table}{0}
\setcounter{figure}{0}

\section{Distance Measures in Cosmology}
Cosmological distances require special considerations as effects of general relativity, which is, expansion of spacetime becomes substantial that objects are very heavily redshifted. Hubble's law, also known as the Hubble–Lemaître law, is the observation in physical cosmology that galaxies are moving away from Earth at speeds proportional to their distance.
\par 
The Hubble constant $ H_0 $ is the constant of proportionality between recession speed $ v $ and distance $ d $ in the expanding Universe;
\begin{equation}
	v = H_0 d
\end{equation}
The subscripted ``0'' refers to the present epoch because in general $ H $ changes with time. The dimensions of $ H_0 $ are inverse time, but it is usually written
\begin{equation}
	H_0 = \SI{100}{\h \kilo \meter \per \second \per \mega \parsec}
\end{equation}
where $ h $ is a dimensionless Hubble constant and is often introduced to denote the ``present'' cosmology. We further define the Hubble distance as
\begin{equation}
	d_H \equiv \dfrac{c}{H_0} = \SI{9.26e25}{\per \h \metre}
\end{equation}
Then there are the density parameters $ \Omega_r $ (radiation density), $\Omega_m$ (matter density), $\Omega_{\Lambda}$ (dark energy density) and $\Omega_k$ (curvature density) which together define the dimensionless Hubble parameter $ E(z) $. All these quantites can be expressed as function of redshift $ z $ and summarised in the following equation;
\begin{equation}
	E (z) = \dfrac{H(z)}{H_0} = \sqrt{\Omega_r(1+z)^4 + \Omega_m(1+z)^3 + \Omega_k(1+z)^2 + \Omega_{\Lambda}}
\end{equation} 
All of these form the basis of a more convenient way of dealing with distances in cosmology: comoving distances. \textit{Proper distance} roughly corresponds to where a distant object would be at a specific moment of cosmological time, which can change over time due to the expansion of the universe. \textit{Comoving distance} factors out the expansion of the universe, giving a distance that does not change in time due to the expansion of space. Mathematically, comoving distance $ d_C $ is given by
\begin{equation}
	d_C(z) = d_H \int_0^z \dfrac{dz'}{E(z')}
\end{equation}
\section{Implementation}
The first goal is to create a filter to increase the signal-to-noise ratio of the eventual stacking that will be done. The galaxies which are too near the walls of the voids, the wall galaxies are to be removed. For this following steps were taken:
\par 
Using the catalogues made available by \citeauthor{pan_cosmic_2012} (\cite*{pan_cosmic_2012}), data was imported for void centres (using \verb|public_void_catalog.txt|) and void galaxies (using the \newline \verb|maglim_void_galaxies.txt|). To get the coordinates of void centres from here, the columns of $ x $, $ y $, $ z $ distances were used and column $ r $ was used to get void radius. To take the cosmology into account, the radius $ r $ by $ 0.7 $ (in accordance with the data). To find the comoving distances to void galaxies and void centres, we use $ d = \sqrt{x^2 + y^2 + z^2} $ where $ (x, y, z) $ denote coordinates. The obtained comoving distances are now multiplied by $ 0.7 $ to take cosmology into account. After this, a simple program was written using the Astropy package\footnote{\url{http://www.astropy.org}} for Python, which basically calculates distance of every void centre and galaxy pair and returns the galaxy number corresponding to the void number which satisfies a given fractional distance condition (eg. all void galaxies which are at a distance from $ 0.5r $ where $ r $ is the void radius).
\section{The code}
The Python code, which implements the process described in the last section is given here:
\begin{lstlisting}[language=Python, caption=Code for filtering wall galaxies]
	import numpy as np
	from astropy.coordinates import SkyCoord
	from astropy import units as u
	
	# importing void and galaxy data
	data_gal = np.loadtxt('maglim_void_galaxies.txt')
	data_void = np.loadtxt('public_void_catalog.txt')
	
	# stripping the data
	ra_void = data_void[:, 0]
	dec_void = data_void[:, 1]
	ra_gal = data_gal[:, 0]
	dec_gal = data_gal[:, 1]
	
	
	x_void = data_void[:, 2] 
	y_void = data_void[:, 3]
	z_void = data_void[:, 4]
	void_radius = data_void[:, 5]*0.7
	
	x_gal = data_gal[:, 6]
	y_gal = data_gal[:, 7]
	z_gal = data_gal[:, 8]
	
	#following code to check if comoving distances are same as given in catalog
	cmvdist_gal = np.sqrt(x_gal**2+y_gal**2+z_gal**2)*0.7
	cmvdist_void  = np.sqrt(x_void**2+y_void**2+z_void**2)*0.7
	
	#the main part of the code
	for i in range(len(coord_void)):#range(len(coord_void)):
		c2 = SkyCoord(ra=ra_gal*u.degree, dec=dec_gal*u.degree, distance=cmvdist_gal*u.Mpc)
		c1 = SkyCoord(ra=ra_void[i]*u.degree, dec=dec_void[i]*u.degree, distance=cmvdist_void[i]*u.Mpc)
		dist = c1.separation_3d(c2)
		frac_dist = dist.value/void_radius[i]
		a = np.where(frac_dist <= 0.5) #this is the case for 0.5r
		print("For void number #",i+1, a[0]+1)
	
\end{lstlisting}
%\baselineskip 24pt


    



